%% ----------------------------------------------------------------------------
%%                              Chinese Abstract
%% ----------------------------------------------------------------------------
\begin{abstract}{无线信道密钥生成,物理层安全,信道互易性}
    无线通信已经在日常生活中发挥着越来越重要的作用,保障无线通信网络的安全具有重要意义。无线通信由于其开放性、脆弱性、拓扑性,极易遭受攻击,目前传统安全机制在无线通信网络安全方面发挥着重要作用。但是传统安全机制拥有明显的局限性:不适用于低功耗的网络节点设备、或被量子计算攻破、密钥分发困难。

    无线通信物理层安全研究为无线通信安全提供了一个新的角度,在无线通信中,通信双方的信道具有良好的短时互易性,因此可以从无线信道中提取出相似信道特征,进而生成一致的会话密钥。无线密钥生成已经成为无线通信安全研究的热点问题,目前已经有大量关于无线密钥生成的理论研究,但是缺乏实际无线密钥生成方案应用的研究。本文基于GNURadio软件无线电开发套件以及通用软件无线电外设(USRP),设计无线密钥生成方案的具体细节,实现TDD/FDD模式下无线密钥生成系统,并使用该系统在室内房间、室内走廊、空旷室外三种不同场景中的终端固定、终端移动以及人员走动三种不同信道环境下长时间测量了无线信道并生成了密钥,并通过CSI相关性、信息泄露率、CSI随机性评估、密钥NIST测试、密钥生成速率、纠错码的纠错性能等相关指标分别评估TDD模式和FDD模式的系统性能,详细分析了不同场景及环境下无线信道密钥生成技术的安全性与可靠性,实验结果表明,通过选择合适的密钥生成参数,都可以在不同的场景及环境下生成满足随机性要求的密钥。本文主要工作以及结论如下。
    \begin{itemize}
        \item 基于GNURadio软件开发套件和USRP通用无线电外设,设计导频信号收发机,并以导频信号收发机为核心搭建完整的无线密钥生成系统。导频信号收发机由导频序列输出模块、数据发射模块、数据接收模块、导频信号检测模块四个模块构建而成,本文设计上层协议来控制四个模块,完成TDD/FDD模式下导频信号的发射和接收,并进一步信道估计、特征量化、信息调和、隐私放大,最终将会话密钥以及相关中间信息转存磁盘以供分析。此外,本文在设计系统时考虑两种调和方案,一种是基于CRC校验码的方案,另外一种是基于纠错码的方案。前者通过CRC校验去除不一致比特,后一种通过异或比特流的方式加密和解密比特流,结合纠错码恢复消息。
        \item TDD模式下,本文使用搭建的无线密钥生成系统,在室内、走廊、室外三种场景中的终端固定、终端移动以及人员走动三种不同信道环境下连续长时间运行,测量了实际环境中连续的无线信道变化并生成密钥,并分析了多项指标。实验结果表明,TDD模式下,合法通信双方CSI之间的互易性远高于和窃听者CSI的互易性,平均CSI信息安全率高达91.01\%。此外,本文通过计算CSI的图像熵观察不同情况下CSI在时域和频域的变化,复杂的信道环境会提高CSI的图像熵。NIST随机测试结果表明,TDD模式下系统生成密钥流在多项NIST测试中具有良好表现,并且本文观察了不同降采样率时密钥通过NIST测试的情况,结果表明,降采样率的提高会带来密钥随机性的提高,但是也会降低密钥生成速率。而在密钥生成速率方面,TDD模式在降采样率为4时高达92.2231 bits/s,而该采样率下的生成密钥流在NIST测试中表现良好。同时,本文还比较了TDD模式下使用BCH和Turbo纠错码进行前向原始密钥比特信息调和的性能,结果表明使用Turbo码进行信息调和明显优于BCH码。
        \item FDD模式下,本文使用搭建的无线密钥生成系统在相同的9种情况下连续长时间测量无线信道并生成密钥,并分析多项指标。实验结果表明,FDD模式下,合法通信双方CSI之间互易性并不完全好于和窃听者的互易性,其平均信息安全率也较差于FDD模式。FDD模式下不同情况下的CSI图像熵与TDD模式相似,但是其生成密钥在NIST随机性测试中的表现稍逊与FDD模式。在密钥生成速率方面,由于FDD模式下密钥一致率的降低,相同的降采样率下,TDD模式的密钥生成速率约为FDD模式的3倍。同时,由于相同的原因,FDD模式下的纠错码调和方案表现较差,在密钥一致率较低时,恢复的消息具有较大的误比特率。
    \end{itemize}
    \quad % 要加一个占位符号,否则item会导致编译不通过
\end{abstract}

%% ----------------------------------------------------------------------------
%%                              English Abstract
%% ----------------------------------------------------------------------------
\begin{englishabstract}{Wireless Channel Key Generation, Physical Layer Security, Channel Reciprocity}
    Wireless communication has played a more and more important role in daily life. It is of great significance to ensure the security of wireless communication network. Wireless communication is vulnerable to attack because of its openness, vulnerability and topology. At present, traditional security mechanisms play an important role in wireless communication network security. However, the traditional security mechanism has obvious limitations: it is not suitable for low-power network node devices, or it is broken by quantum computing, and key distribution is difficult.
    
    The research on physical layer security of wireless communication provides a new perspective for wireless communication security. In wireless communication, the channels of both sides of communication have good short-term reciprocity, so similar channel features can be extracted from the wireless channel, and then a consistent session key can be generated. Wireless key generation has become a hot issue in wireless communication security research. At present, there have been a lot of theoretical research on wireless key generation, but there is lack of practical research on the application of wireless key generation scheme. Based on gnuradio software radio development kit and universal software radio peripheral (USRP), this paper designs the specific details of wireless key generation scheme, realizes the wireless key generation system in TDD / FDD mode, and uses the system in three different scenarios: indoor room, indoor corridor, open outdoor, terminal fixation, terminal movement and personnel walking The system performance of TDD mode and FDD mode is evaluated respectively by CSI correlation, information leakage rate, CSI randomness evaluation, key NIST test, key generation rate, error correction performance of error correction code and other related indexes. The security and reliability of key generation technology of wireless channel in different scenarios and environments are analyzed in detail The results show that the key can be generated in different scenarios and environments by selecting the appropriate key generation parameters. The main work and conclusions are as follows.
    
    \begin{itemize}
        \item Based on gnuradio software development kit and USRP general radio peripherals, the pilot transceiver is designed, and a complete wireless key generation system is built with the pilot transceiver as the core. Pilot signal transceiver is composed of four modules: pilot sequence output module, data transmission module, data receiving module and pilot signal detection module. In this paper, the upper layer protocol is designed to control the four modules to complete the transmission and reception of pilot signal in TDD / FDD mode, further channel estimation, feature quantization, information reconciliation, privacy amplification, and finally session key and phase Turn off the intermediate information transfer to disk for analysis. In addition, two schemes are considered in the design of the system, one is based on CRC check code, the other is based on error correction code. The former uses CRC to remove inconsistent bits, and the latter uses exclusive or bitstream to encrypt and decrypt bitstream, combining with error correction code to recover message.
        \item In TDD mode, this paper uses the wireless key generation system, which runs for a long time in three different channel environments: indoor, corridor, outdoor, terminal fixed, terminal mobile, and personnel walking. It measures the continuous wireless channel changes in the actual environment and generates the key, and analyzes a number of indicators. The experimental results show that in TDD mode, the interaction between CSIS of both sides of legitimate communication is much higher than that of CSI of eavesdropper, and the average CSI information security rate is 91.01 \%. In addition, by calculating the image entropy of CSI, we can observe the change of CSI in time domain and frequency domain under different conditions. The complex channel environment will improve the image entropy of CSI. NIST random test results show that the system generated key stream in TDD mode has good performance in many NIST tests, and this paper observes the key passing the NIST test at different desampling rates. The results show that the increase of desampling rate will improve the randomness of the key, but also reduce the key generation rate. In terms of key generation rate, TDD mode can achieve 92.2231 bits / s at a down sampling rate of 4, and the generated key stream at this sampling rate performs well in NIST test. At the same time, this paper also compares the performance of BCH and turbo error correcting code in TDD mode. The results show that turbo code is better than BCH code in information reconciliation.
        \item In FDD mode, this paper uses the wireless key generation system to measure the wireless channel and generate the key for a long time in the same 9 cases, and analyzes a number of indicators. The experimental results show that the interaction between CSI and eavesdropper is not better in FDD mode, and the average information security rate is lower than that in FDD mode. The CSI image entropy in FDD mode is similar to that in TDD mode, but its key generation performance in NIST randomness test is slightly worse than that in FDD mode. In terms of key generation rate, due to the decrease of key consistency rate in FDD mode, the key generation rate in TDD mode is about 3 times of that in FDD mode at the same sample reduction rate. At the same time, due to the same reason, the performance of FDD mode error correction code reconciliation scheme is poor. When the key consistency rate is low, the recovered message has a large bit error rate.
    \end{itemize}
    \quad % 要加一个占位符号,否则item会导致编译不通过
\end{englishabstract}
