% \chapter{版权信息与更新记录}
% \label{chp:version_license}

% \section{版权信息}

% 本模板基于许元同学于2007年发布的 SEUThesis 和樊智猛同学于2016年发布的 SEUThesix,并在上述工作的基础上增加了一些新特性,并专注于对硕士研究生学位论文的支持。目前该模板能够同时支持学术型硕士研究生和专业型硕士研究生的学位论文。

% ~

% \begin{tabular}{lll}
% 版权所有\copyright 2007--2012    & 许元      &(\url{xuyuan.cn@gmail.com})\\
%                                 & 宋翊涵    &(\url{syhannnn@gmail.com})\\
%                                 & 黄小雨    &(\url{nobel1984@gmail.com})\\
% 版权所有\copyright 2016          & 樊智猛    &(\url{zhimengfan1990@163.com})\\
% 版权所有\copyright 2019--2020    & 宋睿      &(\url{wurahara@163.com})\\
%                                 & 祁欣妤    &(\url{510371665@qq.com})\\
%                                 & 金星妤    &(\url{136204652@qq.com})\\
% \end{tabular}

% ~

% 该程序是自由软件,你可以遵照自由软件基金会发布的《GNU 通用公共许可证条款第三版》来修改和重新发布这一程序,或者 根据您的选择使用任何更新的版本。我们希望发布的这款程序有用,但我们不对其可用性做任何程度的担保,甚至不保证它有经济价值和适合特定用途。更详细的情况请参阅\href{http://www.gnu.org/licenses/gpl.html}{《GNU 通用公共许可证》}。

% 我们基于GPL-v3发布该程序并不代表我们青睐于GPL许可证,相反我们认为GPL许可证是对开源社区的一种威胁和障碍。它如病毒般的传播条款将会极大限制基于GPL协议开发的自由程序的分发与使用。我们使用GPL许可证仅仅是因为我们所基于的程序使用了它,而GPL-v3规定所有对使用了该许可证的程序的二次分发和代码利用都必须使用同样的许可证开放源代码。但本模板的所有开发者和维护者都一致认为有必要声明我们对GPL的厌恶和反对。

% \section{更新历史}

% \begin{description}
%   \setlength{\itemsep}{2pt}
%   \setlength{\parsep}{2pt}
%   \setlength{\parskip}{2pt}
%   \item[3.4.1] 修正了专业型硕士研究生学位论文的相关设定和渲染格式,并在手册中添加了相关说明。
%   \item[3.3.5] 调整了文献引用的格式;调整了BST文件的若干细节,使之符合东南大学研究生院参考文献引用标准;调整了取消链接着色后的边框显示。
%   \item[3.3.3] 将一些专有名词从CLS文件中抽出并放置于CFG文件中,调整了CFG文件的结构;修正了论文A3封面书脊中西文混排时西文基线高度偏低的问题,并在手册中添加了相关介绍。
%   \item[3.3.1] 添加了对专业型硕士研究生学位论文的支持;调整了表格框线的线型和边距。
%   \item[3.2.5] 添加了对模板参数的介绍;添加了对子图的支持。
%   \item[3.1.1] 删去了CLS文件的一些暴露参数;添加了针对Windows操作系统的编译脚本;撰写文档声明,并正式开放源代码。
%   \item[3.0.3] 调整了参考文献渲染格式,使其符合GB/T 7714-2015国家标准。
%   \item[3.0.1] 大幅调整了CLS文件的结构与布局,取消了对博士研究生学位论文的支持。
% \end{description}
\chapter{总结展望}

\section{本文工作总结}

在军事和民用数据传输中,无线通信网络扮演着重要的角色,因此无线通信网络安全研究是一个备受关注的课题。由于无线网络的开发性、脆弱性和拓扑性,无线网络极易收到攻击。目前无线网络安全机制依赖于传统密码学,但依旧存在诸多安全性问题。传统的安全机制依赖第三方机构、不适用于低功耗设备,并且可能会被量子计算攻破。物理层安全为无线通信安全提供了一个新的角度,成为无线通信安全研究中一个新的领域。传统密钥分发通过上层协议保证无线网络的安全性,但是缺乏对物理层的保护。物理层安全直接在物理层设计协议并分发密钥,从根本上解决无线网络安全性问题。

目前基于物理层安全机制的无线密钥生成理论研究较多,基于实际无线密钥生成系统的设计和实现较少,本文设计TDD模式和FDD模式下低时延无线信道密钥生成系统,并基于本文设计的密钥生成系统采集大量实测数据,并通过CSI相关性、信息泄露率、随机性评估、密钥生成速率、纠错码的纠错性能等相关指标分别评估TDD模式和FDD模式的系统性能。

本文首先介绍无线密钥生成研究的理论基础,接着详细介绍了无线密钥生成系统的设计和实现细节,最后提出多个指标量化系统性能。本文设计的无线密钥生成系统基于GNURadio无线软件电开发套件,使用USRP通用无线外设构建导频信号收发机,基于导频信号收发机在TDD模式和FDD模式下发射和接收导频信号,并进一步根据已知导频信号进行信道估计,再通过特征量化、信息调和等步骤协商得到最终的会话密钥。

本文使用该系统在TDD模式和FDD模式下采集了9中情况下的多组信道数据,并基于实测数据计算出性能指标。分析结果表明本系统在TDD模式下各项指标均优于FDD模式,其根本原因是TDD模式下信道互易性更好。TDD模式下,Alice与Bob之间的CSI互相关系数远远高于Alice与Eve之间的CSI互相关系数,不同场景的平均信息安全率在67.05\% ~ 91.01\%。相对来说,FDD模式下,Alice与Bob之间的CSI互相关系数较差,并且不同场景的平均安全率也小于TDD模式。本文还通过图像熵和NIST测试评估系统的随机性,通过图像熵计算结果展示了不同场景下CSI在时域和频域上变化的随机程度,根据NIST测试标准高计算了不同降采样率下生成密钥比特流的随机性,计算结果表明,TDD模式下生成密钥流具有良好的随机性,但是FDD模式下密钥流随机性较差。本文通过数据采集实验,比较TDD模式和FDD模式下的密钥生成速率,实验结果表明,TDD密钥生成速率远高于FDD模式,数据表明,在相同的降采样率下,TDD模式的密钥生成速率大概是FDD模式的3倍。此外,本文比较了BCH码和Turbo码在本系统中的性能,计算结果表明,Turbo码的误码率稍低于BCH码,并且实验表明,FDD模式下,纠错码性能远远差于TDD模式。

\section{未来研究展望}

本文基于GNURadio软件无线电开发套件,设计无线密钥生成系统方案和搭建一套完整的TDD/FDD模式下的无线密钥生成系统,并使用本文设计的无线密钥生成系统采集大量数据,提出多组指标衡量无线密钥生成系统性能,验证无线密钥生成方案的可靠性和安全性。但是本文所实现无线密钥生成系统仍然有许多局限性。

首先,根据实际实验结果,TDD模式下无线密钥生成系统测试数据在多个指标下表现良好,但是FDD模式下的实测数据表现较差,其本质原因是FDD下无线信道互易性较差。因此如何在FDD模式下提高无线信道互易性是一个具有研究意义的课题。

另外,本文在系统的信道探测阶段仅仅做了一次探测,实际上,一次探测会生成一组CSI并根据该组CSI估计信道。因此,在未来阶段的工作,可以在信道探测阶段做多次探测,将每次信道估计结果取平均,以此得到更加准确的信道估计值。

最后,虽然TDD模式下密钥一致率高,因此TDD模式下纠错码性能表现较高。但是FDD模式下,密钥一致率,导致纠错码性能较差,所以通信双方不会得到相同的会话密钥,因此,接收方在对加密之前密钥再附加一组摘要,接收方将纠错之后的密钥计算得到摘要,如果不同,则需要重新协商会话密钥。